%%%%%%%%%%%%
% ABSTRACT %
%%%%%%%%%%%%

\begin{abstract}
   % The ABSTRACT is to be in fully-justified italicized text, 
   % between two horizontal lines,
   % in one-column format, 
   % below the author and affiliation information. 
   % Use the word ``Abstract'' as the title, in 9-point Times, boldface type, 
   % left-aligned to the text, initially capitalized. 
   % The abstract is to be in 9-point, single-spaced type.
   % The abstract may be up to 3 inches (7.62 cm) long. \\
   % Leave one blank line after the abstract, 
   % then add the subject categories according to the ACM Classification Index 
This study introduces the Misclassification Likelihood Matrix (MLM) as a novel tool for quantifying the reliability of neural network predictions under distribution shifts. The MLM is obtained by leveraging softmax outputs and clustering techniques to measure the distances between the predictions of a trained neural network and class centroids. By analyzing these distances, the MLM provides a comprehensive view of the model's misclassification tendencies, enabling decision-makers to identify the most common and critical sources of errors. The MLM allows for the prioritization of model improvements and the establishment of decision thresholds based on acceptable risk levels. The approach is evaluated on the MNIST dataset using a Convolutional Neural Network (CNN) and a perturbed version of the dataset to simulate distribution shifts. The results demonstrate the effectiveness of the MLM in assessing the reliability of predictions and highlight its potential in enhancing the interpretability and risk mitigation capabilities of neural networks. The implications of this work extend beyond image classification, with ongoing applications in autonomous systems, such as self-driving cars, to improve the safety and reliability of decision-making in complex, real-world environments.
%-------------------------------------------------------------------------
%  ACM CCS 1998
%  (see https://www.acm.org/publications/computing-classification-system/1998)
% \begin{classification} % according to https://www.acm.org/publications/computing-classification-system/1998
% \CCScat{Computer Graphics}{I.3.3}{Picture/Image Generation}{Line and curve generation}
% \end{classification}
%-------------------------------------------------------------------------
%  ACM CCS 2012
%The tool at \url{http://dl.acm.org/ccs.cfm} can be used to generate
% CCS codes.
%Example:
\begin{CCSXML}
<ccs2012>
<concept>
<concept_id>10010147.10010257.10010293.10010319</concept_id>
<concept_desc>Computing methodologies~Machine learning</concept_desc>
<concept_significance>300</concept_significance>
</concept>
<concept>
<concept_id>10010147.10010178.10010187</concept_id>
<concept_desc>Computing methodologies~Computer vision</concept_desc>
<concept_significance>300</concept_significance>
</concept>
<concept>
</ccs2012>
\end{CCSXML}


\ccsdesc[300]{Computing methodologies~Machine learning}
\ccsdesc[300]{Computing methodologies~Computer vision}



\printccsdesc   
\end{abstract}  