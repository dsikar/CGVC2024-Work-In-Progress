%%%%%%%%%%%%%%%
% CONCLUSIONS %
%%%%%%%%%%%%%%%

\section{Conclusions and Future Work}

In this study, we proposed a novel approach for quantifying the reliability of neural network predictions under distribution shifts by leveraging clustering techniques and analyzing the distances between softmax outputs and class centroids. 

The introduction of the Misclassification Likelihood Matrix (MLM) provides a comprehensive view of the model's misclassification tendencies, enabling decision-makers to identify the most common and critical sources of errors.
The results highlight the potential of MLMs in enhancing the interpretability and risk mitigation capabilities of softmax probabilities.


The implications of our work extend beyond image classification, with ongoing applications in autonomous systems, such as self-driving cars, to improve the safety and reliability of decision-making in complex, real-world dynamic environments. 
By identifying scenarios where human judgment is preferable to the autonomous system's assessment, our methodology aims to mitigate the risks associated with distribution shifts and enhance the overall trustworthiness of automated decision-making systems.

Future directions: 

1. Incorporate entropy and entropy apex metrics into our framework. The entropy apex, a point approximately equidistant to all class centroids, is expected to be a region of high entropy where misclassified examples cluster. 

2. Explore alternative distance metrics, such as cosine similarity, to measure the relationships between class centroids and predictions. This could offer new perspectives on the geometric properties of the feature space and how they relate to classification decisions.

3. Integrate our metric with complementary techniques, including uncertainty estimation and domain adaptation methods. This integration could lead to a more comprehensive framework for handling distribution shifts, combining the strengths of multiple approaches to create more robust and adaptable models.

4. Investigate the application of our methodology to a broader range of real-world autonomous systems, focusing on how it can enhance safety and reliability in diverse and challenging environments.