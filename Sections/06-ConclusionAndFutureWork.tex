%%%%%%%%%%%%%%%
% CONCLUSIONS %
%%%%%%%%%%%%%%%

\section{Conclusions and Future Work}

In this study, we proposed a novel approach for quantifying the reliability of neural network predictions under distribution shifts by leveraging clustering techniques and analyzing the distances between softmax outputs and class centroids. The introduction of the Misclassification Likelihood Matrix (MLM) provides a comprehensive view of the model's misclassification tendencies, enabling decision-makers to identify the most common and critical sources of errors.
The results highlight the potential of MLMs in enhancing the interpretability and risk mitigation capabilities of softmax probabilities.
The implications of our work extend beyond image classification, with ongoing applications in autonomous systems, such as self-driving cars, to improve the safety and reliability of decision-making in complex, real-world environments. By identifying scenarios where human judgment is preferable to the autonomous system's assessment, our methodology aims to mitigate the risks associated with distribution shifts and enhance the overall trustworthiness of automated decision-making systems.
Future work will focus on several key areas. First, we plan to extend our approach to more complex datasets and network architectures to further validate its effectiveness and generalizability. Second, we aim to investigate the integration of our metric with other techniques, such as uncertainty estimation and domain adaptation, to develop a more comprehensive framework for handling distribution shifts. Third, we will explore the application of our methodology to real-world autonomous systems, such as self-driving cars, to assess its impact on safety and reliability in dynamic environments.
Additionally, we plan to conduct user studies to evaluate the interpretability and usefulness of MLMs for decision-makers in various domains. By gathering feedback from domain experts, we can refine our approach to better meet the needs of practitioners and ensure its practical applicability.
Finally, we aim to investigate the potential of our approach in other areas, such as medical diagnosis, financial risk assessment, and natural language processing, where the consequences of misclassifications can be severe. By adapting our methodology to these domains, we hope to contribute to the development of more reliable and trustworthy AI systems across a wide range of applications.