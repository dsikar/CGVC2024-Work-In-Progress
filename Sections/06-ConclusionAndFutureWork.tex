%%%%%%%%%%%%%%%
% CONCLUSIONS %
%%%%%%%%%%%%%%%

\section{Conclusions and Future Work}

In this study, we proposed a novel approach for quantifying the reliability of neural network predictions under distribution shifts by leveraging clustering techniques and analyzing the distances between softmax outputs and class centroids. 

The introduction of the Misclassification Likelihood Matrix (MLM) provides a comprehensive view of the model's misclassification tendencies, enabling decision-makers to identify the most common and critical sources of errors.
The results highlight the potential of MLMs in enhancing the interpretability and risk mitigation capabilities of softmax probabilities.


The implications of our work extend beyond image classification, with ongoing applications in autonomous systems, such as self-driving cars, to improve the safety and reliability of decision-making in complex, real-world dynamic environments. 
By identifying scenarios where human judgment is preferable to the autonomous system's assessment, our methodology aims to mitigate the risks associated with distribution shifts and enhance the overall trustworthiness of automated decision-making systems.
% We aim to incorporate to our metric, entropy and entropy apex - a point of high entropy approximately equidistant to all class centroids - proximity, where misclassified examples are expected to cluster. Explore additional distance metrics e.g. cosine similarity between class centroids and predictions. Integrate our metric with other techniques, such as uncertainty estimation and domain adaptation, to develop a more comprehensive framework for handling distribution shifts. 
Future directions: 

1. Incorporate entropy and entropy apex metrics into our framework. The entropy apex, a point approximately equidistant to all class centroids, is expected to be a region of high entropy where misclassified examples cluster. 

2. Explore alternative distance metrics, such as cosine similarity, to measure the relationships between class centroids and predictions. This could offer new perspectives on the geometric properties of the feature space and how they relate to classification decisions.

3. Integrate our metric with complementary techniques, including uncertainty estimation and domain adaptation methods. This integration could lead to a more comprehensive framework for handling distribution shifts, combining the strengths of multiple approaches to create more robust and adaptable models.

4. Investigate the application of our methodology to a broader range of real-world autonomous systems, focusing on how it can enhance safety and reliability in diverse and challenging environments.%Explore the application of our methodology to real-world autonomous systems, such as self-driving cars, to assess safety and reliability in dynamic environments.
% Cosine similarity is mentioned in reply to 

% This study introduces a novel methodology for assessing the reliability of neural network predictions under distribution shifts. Our approach leverages advanced clustering techniques and analyzes the distances between softmax outputs and class centroids, culminating in the development of the Misclassification Likelihood Matrix (MLM).

% The MLM offers a comprehensive visualization of a model's misclassification tendencies, providing decision-makers with a powerful tool to identify and prioritize the most frequent and critical sources of errors. Our results demonstrate the significant potential of MLMs in enhancing both the interpretability of model decisions and the capacity for effective risk mitigation strategies based on softmax probabilities.

% While our initial focus has been on image classification, the implications of this work extend far beyond this domain. We envision widespread applications in autonomous systems, particularly in safety-critical areas such as self-driving vehicles. In these complex, dynamic real-world environments, our methodology could substantially improve the safety and reliability of automated decision-making processes.

% A key strength of our approach lies in its ability to identify scenarios where human judgment should take precedence over autonomous system assessments. This capability is crucial for mitigating risks associated with distribution shifts and ultimately enhancing the trustworthiness of automated decision-making systems across various domains.

% Looking forward, we have identified several promising avenues for future research:

% Incorporate entropy and entropy apex metrics into our framework. The entropy apex, a point approximately equidistant to all class centroids, is expected to be a region of high entropy where misclassified examples cluster. 

% Explore alternative distance metrics, such as cosine similarity, to measure the relationships between class centroids and predictions. This could offer new perspectives on the geometric properties of the feature space and how they relate to classification decisions.

% Integrate our metric with complementary techniques, including uncertainty estimation and domain adaptation methods. This integration could lead to a more comprehensive framework for handling distribution shifts, combining the strengths of multiple approaches to create more robust and adaptable models.

% Investigate the application of our methodology to a broader range of real-world autonomous systems, focusing on how it can enhance safety and reliability in diverse and challenging environments.

% Through these future directions, we seek to further refine and expand the capabilities of our approach, contributing to the development of more trustworthy and effective AI systems in critical application areas.

% 
% Additionally, we plan to conduct user studies to evaluate the interpretability and usefulness of MLMs for decision-makers in various domains. By gathering feedback from domain experts, we can refine our approach to better meet the needs of practitioners and ensure its practical applicability.
% Finally, we aim to investigate the potential of our approach in other areas, such as medical diagnosis, financial risk assessment, and natural language processing, where the consequences of misclassifications can be severe. By adapting our methodology to these domains, we hope to contribute to the development of more reliable and trustworthy AI systems across a wide range of applications.